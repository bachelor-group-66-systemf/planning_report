\documentclass[12pt,a4paper]{article}
\usepackage[]{inputenc}
\usepackage[T1]{fontenc}
\usepackage{mathptmx}
\usepackage[]{graphicx}
\usepackage{calc}
\usepackage{enumitem}
\usepackage[a4paper, lmargin=0.1666\paperwidth, rmargin=0.1666\paperwidth, tmargin=0.1111\paperheight, bmargin=0.1111\paperheight]{geometry}
\usepackage{parskip}
\usepackage[all]{nowidow}
\usepackage[protrusion=true,expansion=true]{microtype}
\usepackage[english]{babel}

\frenchspacing
\linespread{1.5}
\setlength{\parindent}{0cm}
\usepackage{float}
\usepackage[
citestyle=numeric,
style=ieee,
]{biblatex}
\addbibresource{ref.bib} 
\usepackage{xcolor}

\usepackage[pdftex,linkcolor=black,pdfborder={0 0 0}]{hyperref}
\hypersetup{ 	
    pdfsubject = {},
    pdftitle = {},
    pdfauthor = {}
}

\title{Planning report}

\begin{document}
\maketitle
\section{Background}
\textcolor{gray}{
The background has to contain reasons why the subject chosen is of interest from an
academic perspective and/or from an engineering perspective or, where relevant, from the
perspective of the customer/client. In certain cases, this heading has to include a brief
history of the subject. After reading the background all readers should understand why the
subject is relevant. The following issues should be considered:
What is the subject/problem to be examined? Why has the subject/problem come up? Why
or for whom is it an interesting or relevant subject/problem? Can the specific
subject/problem be related to a more general discussion?
}
\section{Aim}
\textcolor{gray}{
The aim specifies what the project is intended to result in and what type of results will be
achieved. A project can have several different aims that are related to the subjects/problems
presented in the background. In most cases, however, it is appropriate to only have a
general aim, which is then broken down into more detailed parts further on in the process
and essay/report in the bachelor’s thesis.
}
\section{Task}
\textcolor{gray}{
This section is often the most important part of the planning report (and of the final
essay/report). It aims to identify the question(s) to be raised in the project. It is important
that the group does a problem analysis even if a problem (task) is already specified in the
project proposal. The reason is that the real primary problem often differs from the problem
initially proposed by the client/proposer/customer. The problem analysis is also intended to
break down the problem/task into smaller and more detailed sub-problems/sub-tasks, which
can also lead to the formulation of sub-aims. By doing this the students gain a much better
understanding of the different aspects of the problem/task. Without this information it is
impossible to identify what information is needed, what information sources are needed and
what approaches are suitable.
A good problem analysis that identifies sub-problems/sub-tasks and sub-aims often builds
on the use of theories and models from the literature. A review of the literature should
therefore be carried out early on in the process.
}
\section{Scope}
\textcolor{gray}{
The scope has to take up what parts of the problem will be not taken up in the essay/report
and the reason for this. The reasons given for the scope are important.
}
\section{Implementation}
\textcolor{gray}{
How the group has intended to implement the work is their choice of method. In design-
centered projects this may appear to be obvious, but there may also be important choices of
method in this case. Wholly literature-based bachelor’s theses are also feasible, but even a
literature study has to have an ordered and structured work process and methodology.
The method section should also describe how to collect data and how to establish how well
the aim of the project has been fulfilled. In practical projects this can be through
measurements of various types. It can also be through computer simulations. What aspects
are important in order to know whether the aim of the project has been achieved? Data
collection can also be an important part of testing or other evaluation of the product
developed in a design-focused project.
Number of study objects/test cases and how are they selected? Type of investigation
method/test method? How will the data/test results collected be analyzed and presented?
What does the process for the literature work look like?
}
\section{Maybe (Societal and ethical aspects)}
\textcolor{gray}{
In the planning report the group is expected to write a brief text in which the group assesses
whether societal and ethical aspects need to be taken into account and analyzed further in
the essay/report. The group will benefit from using annex 7 as support along with the digital
resources available on the Student Portal’s pages about the bachelor’s thesis.
}
\section{Timetable}
\textcolor{gray}{
This part of the planning report describes what will be done and when it will be done. The
people to be contacted should also be stated here. The dates or at least the weeks when the
students will submit interim reports and the final presentation have to be stated here. The
timetable will obviously be fairly rough to begin with.
It is important to note that the activities in the project cannot be viewed sequentially as
these activities are dependent on one another, which means that there will be a number of
iterations between them. It will only be possible to use the knowledge built up in a good way
by iterating between them. The same thinking also applies to report writing, i.e. the
updating of a section also requires the updating of other sections. Report writing should
therefore be done continuously during the whole of the project.
}
\end{document}