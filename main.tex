\documentclass[12pt,a4paper]{article}
\usepackage[]{inputenc}
\usepackage[T1]{fontenc}
\usepackage{mathptmx}
\usepackage[]{graphicx}
\usepackage{calc}
\usepackage{enumitem}
\usepackage[a4paper, lmargin=0.1666\paperwidth, rmargin=0.1666\paperwidth, tmargin=0.1111\paperheight, bmargin=0.1111\paperheight]{geometry}
\usepackage{parskip}
\usepackage[all]{nowidow}
\usepackage[protrusion=true,expansion=true]{microtype}
\usepackage[english]{babel}
\usepackage{csquotes}
\usepackage{titling}

\frenchspacing
\linespread{1.0}
\setlength{\parindent}{0cm}
\usepackage{float}
\usepackage[
citestyle=numeric,
style=ieee,
]{biblatex}
\addbibresource{ref.bib} 
\usepackage{xcolor}

\usepackage[pdftex,linkcolor=black,pdfborder={0 0 0}]{hyperref}
\hypersetup{ 	
    pdfsubject = {},
    pdftitle = {},
    pdfauthor = {}
}

\title{Creating a functional programming language based on System F}

\author{William Bodin\and Martin Fredin\and Samuel Hammersberg\and Valter Miari\and Victor Olin\and Sebastian Selander}
\date{}

\begin{document}
\maketitle

\begin{center}
    \includegraphics[scale=0.3]{tjolmers.png}
    
    \includegraphics[scale=1.5]{golleborg.png}
\end{center}

\newpage

\section*{Glossary}
\begin{itemize}
    \item System F - A typed lambda calculus that provides a theoretical basis for mathematical programming languages like Haskell and ML.
    \item Haskell - A general-purpose functional programming language popular in academia.
    \item BNFC - Backus-Naur Form Converter, a tool to generate lexers and parsers in a specified host language. These tools generated will help form the compiler for the language in question. \cite{bnfc}
    \item Lexer - The first part of the compilation phase which reads a given text and splits it into tokens. This phase creates the tokens necessary for the parser to continue the compilation.
    \item Parser - The second phase of compilation, receives tokens from the lexer and parses them into a syntax tree.
    \item AST - Abstract syntax tree, a tree data structure to represent the flow of a program.
    \item Annotated AST - An extension of an AST with type annotations on operations and data types.
    \item LLVM - A compiler suite with reusable tools to create compilers for modern languages. \cite{llvm}
    \item LLVM IR - LLVM intermediate representation, a platform-independent high-level assembly language.
    \item Alex - A tool for generating lexical analyzers in Haskell.
    \item Happy - A parser generator for Haskell.
\end{itemize}

\newpage

\section{Background}
In general, there have always been two distinct branches of programming, imperative and functional programming.
Functional languages have several differences to imperative languages, they come with greater expressive power, a more
declarative style, and a close resemblance to mathematics. The influence of functional programming can also be noticed in
recent trends regarding non-functional languages, some, which have borrowed functionality from functional languages, like C\# implementing Linq
to support more functional styles of programming. Rust is a more recent programming language that has taken a
lot of inspiration from functional languages such as Haskell and OCaml \cite{rustinfl}.

In functional languages, powerful type systems are formed on the basis of mathematics and typed lambda calculus such
as System F \cite{girard1986system}. This provides a way to express the behavior of programs and also catch 
developer-induced bugs already at compile time.

When designing a language, one has to face a critical decision regarding the execution
of programs written in the language in question, whether the language should be
interpreted or compiled. Interpreted languages have the advantage of inheriting the architectural support of the host language. Comparatively, the host language of the compiler, as well as the architecture of the generated code has to be supported. \\
While writing compilers can be tedious for the developer, it comes with the
advantage of the compiled program being faster. When interpreters are executing
a program, the interpreter and its runtime have to handle all communication with the
hardware through the operating system. A compiled program can directly be executed by the 
operating system such that no middle layer exists between the hardware and the program.

LLVM \cite{llvm} is a compiler infrastructure that allows one to compile a program 
to any of its supported target architectures. This compiler infrastructure will be used to 
avoid architecture-specific hardware and software details, such as reading registers and
manipulating the stack. Taking this load off of the developer such that one can
focus more on implementing the language rather than debugging machine code on various
systems.

%All of this can be hard for any developer to grasp completely, even though they are
%important areas of development. That is why it is important to explore these topics
%and to try to learn them because to learn a topic one must try to implement them
%themselves.
%
Clearly, programming language design is a complex task. It is important that different methods are tried and documented, allowing developers in the future to take advantage of the processes that worked well, and also not repeat the mistakes of previous language designers.

\newpage

\section{Aim}

The two primary goals of the project are to explore the process of creating a statically typed functional programming based on System F that is compiled to LLVM \cite{llvm}, and explore some advanced concepts related to programming language theory. As such, two important aspect of the project is documentation as well as verifying the correctness of the implementation. \\
The compiler should be based on good and robust methods, this means that the methods used should, if possible, be based on previous successful attempts. One such example is basing our type rules solely on those of System F.


\section{Scope}
Implementing a useful general-purpose language with all the functionality users have become accustomed to is complex and time-consuming. 
Therefore, scoping, in the sense of; knowing what to prioritize and trying to define what is achievable within a certain time frame, is an integral part of the process.

The language is not supposed to be useful, but rather it is an academic exercise. One example of uselessness is the lack of effects, i.e. interactions with the outside world. The only planned effect is to print the computed value.

Another feature that is out of scope for this project is an extensive standard library. 
Many modern programming languages have a big collection of functions for performing common tasks. 
One example of this is the programming language Go,
which is a programming language initially developed as a systems language with a focus on multi-processing \cite{go}. Still, the first release of Go included a package for creating a web server. 
The language of this project will not focus on a standard library because there seem to be fewer interesting design decisions, in comparison to other parts of the compiler.   


Other areas that will not be implemented:
\begin{itemize}
    \item External library management
    \item Linking and modules
    \item Code optimization
    \item Concurrency
    \item Input/Output
\end{itemize}

\newpage

\section{Implementation}
A compiler consists roughly of the following six phases: lexical analysis, syntactic analysis or parsing,
semantic analysis, intermediate code generation, code optimization, and code generation.

\subsection{Lexer and Parser}
The lexer and the parser will be generated using BNFC, as it provides an ergonomic way of quickly developing and making
changes to the grammar.  If it turns out that the limitations of BNFC are too great, a custom lexer and parser will be
created, perhaps using tools such as Alex and Happy.

\subsection{Type checker}
In the type checker, the abstract syntax tree from parsing is annotated with type information. This part of
the compiler also validates the tree using the type-checking and type inference rules of System F. As
validation is done in the type checker a lot of the syntactically valid programs from parsing are deemed invalid,
thus only a small subset of programs can pass this step. Because of this, the next step of the compiler
becomes a lot cleaner, as no validation is needed.

\subsection{Code generation \& Compilation}
In these steps, the resulting annotated AST from the type checker is sent to 
the code generation function which traverses the tree, creating a new tree
which represents LLVM's intermediate language, LLVM IR.

During this process, additional functionality, like writing text to the user, is baked into 
the program, as this is something that can not be implemented in our language due to the 
required bindings which are out of scope for this project.
This finalized tree can be compiled into machine code using LLVM, which can then be run on
the target platform.

\subsection{Garbage collection}
A garbage collector will be developed together with the language run-time and will be implemented
in a lower-level language such as C++ to leverage its speed and lower-level operations.
The implementation design is yet to be determined, as the project and tool chain structure
are quite complex. One idea is to construct the garbage collector as an interface and invoke
endpoints in LLVM as it supports external functions which do not have to be available
at compilation. The benefit of this idea is that the garbage collector does not have to
be recompiled every time the user compiles their program. However, this makes it such that the garbage collector has to be
target specific, which results in more work for the developer to choose what platforms to support.

There are multiple algorithms for implementing a garbage collector and an
endless amount of combinations. The algorithm used for this project will be determined later
during the project when we are more confident in what suits the language's and the group's needs.

\subsection{Documentation}
As the language evolves the importance of the creation of a language specification and 
language reference grows with it.
As with any proper programming language, having documentation on what the language includes 
and its capabilities
is crucial for curious developers wanting to learn the language, or other developers wanting 
to create their own compiler for the language.

The specification will include the syntax and the grammar for the language, removing any 
ambiguity about what
is a syntactically correct program. It should also include a description of the type system 
and some information
about the compiler and the outputted machine code.

The language reference should act as a manual for the language, showing how to use the 
language and the basic elements such as the operators and the limited standard library.

\subsection{Correctness}
One important part of creating a programming language is verifying its correctness.
The semantics should be sound and follow mathematical laws, and the behavior of the compiled 
code should be consistent and correct.
Proving the correctness of the compiler is a complicated and time-consuming task. The focus 
will instead be properly testing all parts of the compiler. This will be done using unit 
tests in Haskell, as well as writing property-based tests using QuickCheck. \cite{claessen2000quickcheck}

\subsection{Milestones \& success}
The success of the project is determined by how many of the listed milestones are achieved. The baseline of the project is
covered by the first four points. Pattern matching and recursive algebraic data types are extensions to the language that
make it actually practical to use. If time allows, garbage collection and linear types are more advanced concepts that
will be explored.

\begin{itemize}
\item Type checking
\item Type inference
\item Code generation to LLVM IR
\item Code generation to machine code using LLVM bindings
\item Pattern matching
\item Recursive algebraic data types
\item Linear types
\item Garbage collection
\end{itemize}


\section{Societal and ethical aspects}
Creating a programming language is generally not a question of ethics.
The aspects to consider are mostly plagiarism and transparency. In this project, we will adhere to the code of ethics as
defined by the Association of Computing Machinery \cite{ACM}. The most important point in the code of ethics is 1.5
\emph {Respect the work required to produce new ideas, inventions, creative works, and computing artifacts}, which states that
original authors should be credited for their work, as well as respecting licenses and copyrights.

One should also follow point 1.3 \emph{Be honest and trustworthy}, which states that one should be honest about the limits and
capabilities of one's work, and our ability as well. This means we should be honest about flaws and untested parts of our
project, and not exaggerate the capabilities of our language.

\section{Timetable}
\begin{center}
\begin{tabular}{|c|l|l|}
\hline
\textbf{Date} & \textbf{Activity}  & \textbf{Contact} \\ 
\hline
\hline
% 6/2  & Draft of planning report finished before supervision & - \\ 
% 9/2  & Technical language supervision                       & - \\ 
% 10/2 & Hand in planning report                              & - \\ 
13/2 & Start working on the compiler                        & - \\ 
\hline
7/3  & Half time presentation                               & - \\
\hline
12/3 & First individual contribution report                 & - \\
\hline
27/4 & Hand in the first draft of the report                    & Supervisor \\
\hline
12/5 & Finalize the compiler                                & - \\
\hline
15/5 & Hand in the second draft of the report                   & - \\
\hline
15/5 & Contribution report                                  & - \\
\hline
17/5 & Hand in video presentation                           & - \\
\hline
22/5 & Individual written opposition                        & Opposition group \\
\hline
25/5 & Presentation                                         & - \\
\hline
26/5 & Presentation                                         & - \\
\hline
29/5 & Second individual contribution report                & - \\
\hline
2/6  & Hand in final report                                 & - \\
\hline
\end{tabular}
\end{center}

\newpage
\printbibliography
\end{document}